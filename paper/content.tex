% status: 100
% chapter: Gradient Boosting

\title{XGBoost}


\author{Goutham Arra}
\affiliation{%
  \institution{Indiana University Bloomington}
  \streetaddress{804 N Woodbridge Drive}
  \city{Bloomington}
  \state{Indiana}
  \postcode{47408}
  \country{USA}
}
\email{garra@iu.edu}

\author{Gregor von Laszewski}
\affiliation{%
  \institution{Indiana University}
  \streetaddress{Smith Research Center}
  \city{Bloomington} 
  \state{IN} 
  \postcode{47408}
  \country{USA}}
\email{laszewski@gmail.com}


% The default list of authors is too long for headers}
%\renewcommand{\shortauthors}{G. v. Laszewski}


\begin{abstract}  

XGBoost stands for Extreme Gradient Boosting. XGBoost is an open source software
library where you can  implement Gradient Boosting machine learning method. It
was created by Tianqi Chen, now with contributions from many developers and part
of a wider collection of open source libraries developed by Distributed Machine
Learning Community(DMLC).

\end{abstract}

\keywords{hid-sp18-401, Machine Learning, Gradient Boosting}


\maketitle


\section{Introduction}  

Before talking about XGBoost, it is best to give introduction to general
<<<<<<< HEAD
gradient boosting method. Gradient Boosting is a machine learning technique
used to build both regression and classification models that is primarily
based on decision trees(it works on other type of models too). A decision tree
is a type of mapping technique of the data or observations to the target
variable based on a certain splitting criteria, that has yes or no answer,
dividing input data into two groups, therby forming a tree.  Building only a
single decision tree model on a large dataset, has been proven to be  unstable
(has large variance) for making any prediction. This gave birth to a type of
model building called ensemble models, which is stacking of decision trees
together, forming an additive model of them that is much more powerful in
reducing the overall variance. In simple words, an ensemble model is weighted
aggregation of many simpler models, called decision stumps. Instead of
stacking all the models together, Boosting improves model accuracy by
optimizing  the hyperparemeters that are involved. One of such Boosting
techniques is Gradient Boosting, which involves differentiation of the loss
function, to find the optimum point, and then updating the  parameters
accordingly, for a fixed number of iterations or until the loss converges
below a threshold point.

But building gradient boosting models on huge datasets (that sometimes contain 
more than 500,000 observations) is computationally onerous, not so efficient.
By the use of XGBoost, we can build  heavy gradient boosting models on large
data sets and run them fastly with better performance. ``The name xgboost,
though, actually refers to the engineering goal to push the limit of
computations resources for boosted tree algorithms. Which is the reason why
many people use xgboost.'' - says Tianqi Chen, creator of XGBoost ~\cite {hid-
sp18-401-XGBoost-MLmastery}. The description of XGBoost according to the
software repository on github is ``Scalable, Portable and Distributed Gradient
Boosting (GBDT, GBRT or GBM) Library, for Python, R, Java, Scala, C++ and
more'' ~\cite{hid-sp18-401-XGBoost-Github}.
=======
gradient boosting method. Gradient Boosting is a machine learning technique used
to build both regression and classification models. It is primarily used in
building decision trees. But building gradient boosting models on huge datasets
(that sometimes contain more than 500,000 observations) is computationally
onerous, not so efficient. By the use of XGBoost, we can build  heavy gradient
boosting models on large data sets and run them fastly with better performance.
``The name xgboost, though, actually refers to the engineering goal to push the
limit of computations resources for boosted tree algorithms. Which is the reason
why many people use xgboost.'' - says Tianqi Chen, creator of XGBoost ~\cite
{hid-sp18-401-XGBoost-MLmastery}. The description of XGBoost according to the
software repository on github is ``Scalable, Portable and Distributed Gradient
Boosting (GBDT, GBRT or GBM) Library, for Python, R, Java, Scala, C++ and more''
~\cite{hid-sp18-401-XGBoost-Github}.

>>>>>>> ba5e3e9c1a3f88807cb7f9d8daa88c0736d89ca8


\section{How XGBoost Model Works}

XGBoost model is just like any other Gradient Boosting model, in that it
optimizes a loss function  along with regularization term. But it has its own
implementation, with computations done on special kind of input format called
DMatrix, along with a lot more customizable options for the model.
 
For an ensemble model, which is a aggregation of a number of decision trees,
there exists an objective function also called a loss function, that
calculates the difference in score between predicted values with that of
actual score. Objective function also includes a second term called
Regularization that helps in preventing overfitting our model to the training
data. Regularization is a key aspect of XGBoost, that makes our model more
robust to increased variance. In XGBoost, we use gradient descent method to
optimize our loss function or objective function. In this method, we
differentiate our loss function with respect to predicted output (y-hat),
which is the gradient, and improve this 'y-hat' along the direction of
gradient to minimize the loss function in an iterative manner.

XGBoost model supports the features of scikit-learn (in python) and R
implementations, like cross validation. fit methods etc. along with new
additions like regularization.  Typical loss functions that one uses depend on
the type of Machine Learning model that we are building. Some of them are :

\begin{itemize}
\item Root Mean Square error for Linear Regression problems. 
\item Log loss for Binary Classification problems. 
\item mLog loss for Multi Class Classification problems.
\end{itemize}

XGBoost optimizes its parameters by using Gradient Descent Algorithm on the loss
function ~\cite{hid-sp18-401-XGBoost-MLmastery}. Three forms of gradient
boosting  are supported by XGBoost, which are:

\begin{itemize}
\item Gradient Boosting(regular)
\item Stochastic Gradient Boosting
\item Regularized Gradient Boosting 
\end{itemize}

\section{Evolution of XGBoost} 

The creator of XGBoost, Tianqi Chen, is well versed in large-sccale machine
learning and wanted to build scalable learning systems. XGBoost was one such
project. It was created out of necessity. While Tianqi was doing research on
variants of tree boosting, he did not find any existing Machine Learning
method that satisfied the  combination of boosted trees and conditional random
field. This gave birth to the first version of XGBoost, which was nothing but
a configuration file. To use this model,  one has to use this config file,
along with parameters and read the data for the model in LibSVM format. When
Tianqi learned the popularity of XGBoost in the Higgs Boson ocmpetition, he
along with his friend Bing Xu, to build a prototype XGBoost model with python
wrapper, and later another one with R package. According to Tianqi, a machine
learning model cannot be succesful alone with XGBoost. For a model to be good,
all other things like data pre-processing, feauture engineering etc. had to be
factored., which is why, "interfacing XGBoost with R and Python enables
XGBoost to naturally work with language native data pipelines and become much
more powerful~\cite{hid-sp18-401-XGBoost-evolution}." It also helps in easy
customization and allows user to better use the model for better results. That
is why, it did not take long for XGBoost to be integrated with scikit-learn
package in Python package, which became more powerful, now that users can use
grid search feature to optimize the hyperparameters of the model. Next thing
left, is to make XGBoost more scalable on distrubuted systems. It was achieved
by building a common Allreduce runtime called Rabit and use it as a bridge to
port to different systems and  has edge over other machine learning models
like for example, Support Vector Machines(SVM).
See the below figure~\ref{f:XGBoost-on-different-platforms} for more
understanding.

\begin{figure}[!ht]
  \centering\includegraphics[width=\columnwidth]{images/XGBoost-on-different-platforms.png}
  \caption{XGBoost Higgs Boson Competettion~\cite{hid-sp18-401-XGBoost-on-different-platforms-image}}
\label{f:XGBoost-on-different-platforms}
\end{figure}

\section{Interfaces}

XGBoost supports various interfaces on your machine after you download and
install the software ~\cite{hid-sp18-401-XGBoost-MLmastery}. Here are a few 
of the main interfaces.

\begin{itemize}
\item Command Line Interface (CLI)  
\item C++  
\item Python interface along with built in model from Scikit-Learn library   
\item R interface along with built in model in the Caret pachage  
\item Julia item Java and JVM languages like Scala and platforms like Hadoop
\end{itemize}



\section{System Features} 

Some of the system features that XGBoost supports are follows: 

\begin{description}
\item[Parallelization]

It has the ability to use all of your CPU cores, as well as GPU cores (if
available) for model construction and training

\item[Distributed Computing]

It uses cluster of machines to train models with very large datasets (more than
500,000) on frameworks like Apache Hadoop, Apache Spark, and Apache Flink.

\item[Out of Core Computing]

When we have a large datasets that dont fit into the computer memory, XGBoost
has the capability to access the data  like from external harddrive or when
memory is on a computer network and perform Computing on them.

\item[Cache Optimization]

XGBoost has the ability of Cache Optimization of data structures and algorithm
to make the best use of hardware.

\end{description}


\section{Algorithm Features} 

XGBoost was engineered with aim of best efficiency of compute time and memory
resources. It is designed to make best use of available resources to train the
model ~\cite{hid-sp18-401-XGBoost-MLmastery}. Here are some of the key algorithm
implementation features:

\begin{description}

\item[Sparse Aware]

Automatic awareness and handling of missing values in the train data before 
building.

\item[Block Structure]

Supports constructing decision trees in parallel therby optimizing compute
time.

\item[Continued Training] 

Continued training implies, we can further boost or train an already trained
model on new data available to us.

\end{description}


\section{Parameter Tuning} 

XGBoost has added advantage over regular Boostedmodels in the number of
parameters  to tune and more options to optimize your model The overall
parameters are divided  into three categories.

\subsection{General Parameters}

These define the overall functionality of XGBoost

\begin{description}

\item[booster[default - gbtree]]

You get to choose whether you want to implement tree based models(gbtree) or
linear  models (gblinear).

\item[silent[default - 0]]

A silent mode of 1 will stop display of any running messages, but it is
generally  advisable to keep silent mode of 0, to understand how your model is
training and  look for any errors.

\item[nthread]

As we know XGBoost supports parallel processing, with this parameter we can
specify number of threads or cores tobe used simultaneously. If we donot specify
any value  all the cores will be used.

\end{description}

\subsection{Booster Parameters} 

We will consider the following tree booster parameters because they always
outperform  the linear booster.

\begin{description}
\item[eta[default = .3]]

This is the learning rate to be used in stochastic gradient descent lose
optimization  technique boosting. This has the effect of making the model more
robust by shrinking the  weights on each training step.

\item[min child weight[default = 1]] 

As we know a decision tree has weights assigned to all the child nodes
(including leaf nodes) for all observations. This parameter defines minimum
threshold for the sum of  weights.This is very much useful to control
overfitting of our boosted model  (Higher  value implies more overfitting).

\item[max depth[default = 6]]

This is simply the maximum depth of a decision tree, same as in normal gradient
boosting  algorithm. Also used to control overfitting. We tune this parameter by
using cross validation.

\item[max leaf nodes]

This is the maximum number of terminal nodes or leafs in a decision tree.

\item[gamma[default=0]]

In a decision tree a node is split only if it results in decreased Entropy or
increased  Information Gain. Through this parameter  we can specify minimum
threshold for this positive  reduction in the loss function.

\end{description}

\subsection{Learning Task Parameters} 

This parameters are related to optimization objective and the metric to be
calculated at each training step.

\begin{description}

\item[objective[default = reg:linear]]

Based on the type of machine learning model we are building, regression or
classification (binary or multi class) we specify our objective or loss
functions as reg:linear or binary:logistic or  multi:softmax respectively.

\item[eval metric[default according to objective]

This is the metric that is optimized according to the loss function while doing
cross  validation on the train data. The default values are rmse for linear
regression and  error  for classification. We can also use area under curve(auc)
for classification.

\end{description}

<<<<<<< HEAD
\section{Use Cases}

\subsection{Higgs Boson Competition}
=======
\section{Case Study - Higgs Boson Competition}
>>>>>>> ba5e3e9c1a3f88807cb7f9d8daa88c0736d89ca8

Higgs Boson was a famous Kaggle Machine learning challenge in the year 2014,
which has the  aim of classifiying the decaying events of Higgs Boson particle
into tau-tau decay or simple  background noise. The XGBoost made its debut in
Higgs Boson Competition. At the beginning of  the competition, Tianqi,creator
of XGBoost, introduced the tool along with a benchmark code,  which acheived
top 10 percent already. Towards the end of the competition, it was the most
used tool by many ~\cite{hid-sp18-401-Kaggle-Higgs-Boson}. 

The challenge is to build a Classification model that will classify the test
events into either tau-tau decay or background noise. Logistic Regression
machine learning model is known for giving better results on binary
classification tasks. But XGBoost outperformed all such models, and has gained
a lot of popularity with this competition. The training data consisits of
250000 events and test data consists of 550000 events, which is perfect for
XGBoost to handle. There are some missing values in the dataset, which  have
the value -999, but there is a special feature in XGBoost where you can
mention the missing value format that considers them as missing data while
reading input data into DMatrix. DMatrix is input format on which XGBoost
performs computation tasks, which is the main reason for its efficiency, and
computation speeds.

The author and his team in the citation, had built their model based on single
XGBoost instance, which achieved 11th on the leaderboard, which is more than
impressive considering there were 1785 teams in total. They have built their
model with decision trees of depth 6 levels, and 'binary:logitraw' as the
objective function .Besides, giving good performance on the test data, the
efficiency is also very high with XGBoost. Run time of the XGBoost model is
1/4th to that of the usual python sklearn model. We can see the comparision in
the figure~\ref{f:XGBoost-runtime-comparision}. With some basic feature
engineering and minimal parameter tuning, one can achieve around 25th position
on the leaderboard with a single XGBoost model ~\cite{hid-sp18-401 -XGBoost-
pdf}.

\begin{figure}[!ht]
  \centering\includegraphics[width=\columnwidth]{images/Runtime-vs-Number-of-trees.png}
  \caption{XGBoost Higgs Boson Competettion~\cite{hid-sp18-401-XGBoost-runtime-comparision-image}}
\label{f:XGBoost-runtime-comparision}
\end{figure}

\subsection{Porto Seguro's Safe Driver Prediction Challenge} 

This is Kaggle Machine learning challenge in the year 2017, that aims at
reducing vehicle insurance for good drivers and increasing the same for bad
drivers. Given some data about a particular driver, we should predict whether
he will initiate an auto insurance claim for next year or not. By this, the
Auto insurance companies  will have better understanding about their
customoers, by fixing driver-specefic premiums, which is aimed at improving
reputation for their company ~\cite{hid-sp18-401-Kaggle-Porto- Seguro}.

This is a data heavy challenge, consisting of 595,212 train observations and
892,816 test observations with 58 columns or features. Attempting to build a
complex ensemble model and boosting it will be computationally infeasible and
will not produce great results. The following testimony on the benefit of
using XGBoost for this challenge comes from my own experience. Firstly, I had
cleaned up my dataset, filling all the missing values and had done some one-
hot encoding for categorical variables. Before knowing about XGBoost, as with
any other binary classification problem, I have built my models based on
Random Forest Regressor, another Machine learning technique on decision trees
that works with the principle of Bagging. I have achieved a very low
normalized gini coefficient of 0.26. With the help of feature importances of
Random Forest Regressor, I have eliminated unimportant variables from my
dataset. After reading many kernels from many teams, I have learned about the
powerful tool, that is, XGBoost. I have thus built an ensemble model, based on
XGBoost, on the modified dataset and had achieved a much improved gini
coefficient of 0.29 on the test dataset. Also, running this boosted model on
the huge data set took less time than other models that I have tried. This
really proves, how XGBoost is not only a better model but also computationally
efficient.

\section{Conclusion} 

XGBoost is an advanced implementation of normal gradient boosting algorithm
which is easy to use, efficient, has better accuracy and feasible. Its
advantages lie in features such as regularization, parallel processing, high
flexibility in parameter tuning, handling missing values, tree pruning,
built-in cross validation etc, which is why it has become go-to Machine
learning method for several Kaggle competitions. With the advent of XGBoost,
other software implementations of Gradient Boosting are coming up for example,
LightGBM (by Microsoft).

\begin{acks}

  The authors would like to thank Dr.~Gregor~von~Laszewski for his
  support and suggestions to write this paper.

\end{acks}

\bibliographystyle{ACM-Reference-Format}
\bibliography{report} 

